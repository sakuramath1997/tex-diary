\subsection{1章}
\subsubsection{本文中の演習問題}
\subsubsection{章末の演習問題}
\begin{exercise}
    可換環\(\CommutativeRing\)の冪零元\(\Element\)につき,\(1+\Element\)は\(\CommutativeRing\)の単元であることを示せ。
    また,可換環に於いて,冪零元と単元との和は単元であることを示せ。
\end{exercise}
\begin{answer}
    \(\Element\)は冪零元なので正の整数\(\NaturalNumber\)が存在して\(\Element^{\NaturalNumber}=0\)が成り立つ。
    よって
    \begin{center}
        \((1-\Element)(1+\Element+\Element^2+\cdots+\Element^{n-1})=1-\Element^{n}=1\)
    \end{center}
    が得られ,\(1-\Element\)は単元である。
    特に\(-\Element\)を\(\Element\)として取り直せば,\(1+\Element\)が単元であることが分かった。

    単元\(\Element<u>\)を任意に取ると,\(\Element<u>+\Element=\Element<u>(1+\Element<u>\Inv\Element)\)と分解でき,
    \(\Element<u>\Inv\Element\)は冪零元である。
    よって先に示したことから\(1+\Element<u>\Inv\Element\)が単元であること,即ち,\(\Element<u>+\Element\)が単元であることが得られた。
\end{answer}
\begin{remark}
    後半の証明に於いて\(\Element<u>\Inv\Element\)が冪零元であることを用いたが,これは非可換環だと成り立つのかは考えられていない。
\end{remark}

\begin{exercise}
    
\end{exercise}