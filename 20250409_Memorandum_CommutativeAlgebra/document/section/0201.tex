\section{局所環付空間}

%組\( \OrderedSet< \TopologicalSpace, \StructureSheaf > \)が環付空間であるとは,
%\begin{definition}
%    \(\TopologicalSpace\)は位相空間であって,
%    \(\StructureSheaf\)は\(\TopologicalSpace\)の上の環の圏に値を取る層であることをいう。
%    環付空間が局所環付空間であるとは,
%    各点の茎が局所環であることをいう。
%\end{definition}
%\begin{definition}
%\( \RingedSpace \)および\( \RingedSpace<Y> \)を環付空間とする。
%
%    組\( \OrderedSet<Map, \Pushed(\Map)> \)が\( \RingedSpace \)から\( \RingedSpace<Y> \)への射であるとは,
%    \( \Map \)が\( \UnderlyingSpace(\RingedSpace) \)から\( \UnderlyingSpace(\RingedSpace<Y>) \)への位相空間の射であり,
%    \( \Pushed(\Map) \)が\( \StructuredShead_{\RingedSpace<Y>} \)から\( \Pushed(\Map)(\StructuredSheaf_{\RingedSpace}) \)への層の射であることをいう。
%    局所環付空間の間の環付空間の射が局所環付空間の射であるとは,
%    茎に誘導される射が局所環の射であることをいう。
%
%\end{definition}
%\begin{definition}
%
%
%\end{definition}
