\section{Gorensteinホモロジー代数}

本節は,\cite{becerril2019relativegorensteinobjectsabelian}の導入を参考にしています。

1969年に\WordAuslander-\WordBridger\cite{auslander1969stable}に於いて\WordCommutativeNoetherianRing\( \CommutativeNoetherRing \)上の有限生成加群の圏に対して\WordGorensteinDimension を導入しました。
彼らは任意の有限生成\( \CommutativeNoetherRing \)-加群\( \Module \)について不等式\( \GorensteinDimension(\Module) \leq \ProjectiveDimension(\Module) \)が成立すること,更に\( \ProjectiveDimension(\Module) \)が有限であるとき,またそのときに限り,等号が成立することを示しました。
この不等式は\WordGorensteinLocalRing の特徴付けや,\WordGorensteinDimension の場合に\WordAuslanderBuchsbaumFormula を一般化する際に用いられます。

1990年代の中頃には\WordGorensteinDimension の概念は\WordCommutativeNoetherianRing 上の有限生成加群の枠組みを超えて一般化されます。
任意の環\( \Ring \)の上の任意の加群\( \Module \)に対して,\WordEnochs と\WordJenda は%\cite{JendaEnochs1995}
に於いて\WordGorensteinProjectiveDimension\( \GorensteinProjectiveDimension(\Module) \)を定義しました。
その後,\WordAvramov,\WordBuchweitz,\WordMartsinkovsky,\WordReiten らは,未出版の論文``Stable Cohomological Algebra''に於いて\WordNoetherianRing 上の有限生成加群\( \Module \)が\( \GorensteinDimension(\Module) = 0 \)であるとき,またそのときに限り,\WordGorensteinProjective であることを示しました。

近年,\WordHolm は%\cite{JendaEnochs1995}
で研究された\WordGorensteinModule 全体の為すクラスが\WordResolvingClass であることを示し,これによって左\( \Ring \)-加群の圏\( \ModCat(\Ring) \)に於いて\WordRelativeHomologicalAlgebra を適用できるようになりました。
この結果を嚆矢として,\WordGorensteinModule の様々な一般化が考えられるようになりました。
例えば,\WordBravo-\WordGuillespie-\WordHovey は\WordACGorensteinProjectiveModule を定義し,
\WordDing-\WordLi-\WordMao は\WordDingProjectiveModule を定義しました。
これに限らず多くの研究者が\WordGorensteinObject を定義しています。

このような流れにあり,\WordBecerril-\WordMendoza-\WordSantiago\cite{becerril2019relativegorensteinobjectsabelian}に於いて\WordGorensteinObject を統一的に取り扱う手法として\WordRelativeGorensteinObject の概念を導入しました。
\WordRelativeGorensteinObject に関する諸性質を調べる際には,\WordAuslanderBuchweitzApproximationTheory を用います。

