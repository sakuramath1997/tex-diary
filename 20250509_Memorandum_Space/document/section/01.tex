\def\NewCategory#1{%
    \expandafter\def\csname #1Cat\endcsname{\mathsf{#1}}%
}
\def\NewWord#1#2{%
    \expandafter\def\csname Word#1\endcsname{#2}%
}

\NewCategory{Top}
\NewWord{CartesianClosed}{Cartesius閉}

\section{導入}
\subsection{位相的構造について}
本副節は\cite{Herrlich__1974__Topological_Structures}を参照しています。

位相空間の概念は,トポロジーに於いて最も考えられてきました。
しかしながら,位相空間論はそれ自体が困難を有しています。
\begin{enumerate}
    \item 位相空間と連続写像の圏\(\TopCat\)は\WordCartesianClosed ではありません。
        即ち,位相空間\(X\),\(Y\)につき,\(Y\)から\(X\)への連続写像全体\(X^Y\)に対して標準的な位相が定め,
        \(Z\)を位相空間とするとき\((X^Y)^Z\)と\(X^{Y\times Z}\)とが自然に同型になるようにはできません。

\end{enumerate}