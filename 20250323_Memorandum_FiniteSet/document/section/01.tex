\section{採用する公理系}
有限性の定義のバリエーションはいろいろ考えられますが,
これらが等価か否かは公理系の選び方によって変わります。
本稿の趣旨は有限性の緻密な関係を観察し,整理することにあるため,
弱い集合論から始めて徐々に公理を強めていくアプローチを取ることにします。
具体的には,\(\CZF\),\(\IZF\),\(\ZF\)の順に考えていき,
可算選択公理を課した\(\ZF\)の下ではすべて等価になることを確かめます。

本節では次節以降の準備のため,構成的集合論の公理系\(\CZF\)および直観的集合論の公理系\(\IZF\)を手短に紹介します。
その際,直観主義一階述語論理\(\IQC\)\footnote{正確にはHilbert流の形式化\(\HIQC\)を想定していますが,フォーマルな証明木を書かないため,\(\NIQC\)であっても問題なく読みうると思われます。}をベースに議論を進めます。
主に\cite{sep:set_theory__constructive_an_intuitionistic_zf}および\cite{sep:axiom_of_czf_and_izf}を参照しています。

\subsection{言語}
