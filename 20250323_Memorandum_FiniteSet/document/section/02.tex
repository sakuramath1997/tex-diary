\section{構成的集合論に於ける有限性}

本節では先ず素朴な有限性の概念を三つ導入し,
それぞれの概念の関係性を構成的数学に於いて観察します。
その中で,決定可能等価性と限定全知原理を導入し,
この条件が有限性の概念の関係を特徴づける際に用いられることを確かめます。
また,本節に於ける議論は,\cite{yotsunva:2024:intro}に依る部分が大きく,
本稿で扱っていない議論も少なくないため,
こちらも併せて参照されることをおすすめいたします。

\subsection{三つの有限性}

\begin{definition}[Bishop有限性]
    集合\(\Set\)がBishop有限であるとは,
    ある自然数\(\NaturalNumber\in\NaturalNumberSet\)が存在して,
    \(\Set\)と\(\NaturalNumber\)との間に全単射写像が存在することをいう.
\end{definition}

\begin{definition}[Kuratowski有限性]
    集合\(\Set\)がKuratowski有限であるとは,
    ある自然数\(\NaturalNumber\in\NaturalNumberSet\)が存在して,
    \(\NaturalNumber\)から\(\Set\)への全射が存在することをいう.
    Kuratowski有限集合は,有限添数付可能(英:finitely indexed)集合ともいう.
\end{definition}

\begin{definition}[部分有限性]
    集合\(\Set\)が部分有限(英:subfinite)であるとは,
    ある自然数\(\NaturalNumber\in\NaturalNumberSet\)が存在して,
    \(\Set\)から\(\NaturalNumber\)への単射が存在することをいう.
\end{definition}
これらの概念の間には,次のような自明な含意関係が成り立ちます。    
\begin{center}
    \begin{tikzpicture}
        \node (K) at (0,0.7) {\(\text{(Kuratowski有限)}\)};
        \node (S) at (0,-0.7) {\(\text{(部分有限)}\)};
        \node (B) at (-4,0) {\(\text{(Bishop有限)}\)};
        \draw[->, thick, double, double distance=1pt, bend left=8] 
            (B) to (K);
        \draw[->, thick, double, double distance=1pt, bend right=10] 
            (B) to (S);
    \end{tikzpicture}
\end{center}
これは,Bishop有限性の条件の証拠子(英:witness)である全単射写像は全射であるため,それ自体がKuratowski有限性の条件の証拠子となるからです。
Bishop有限性が部分有限性を導くことも同様に分かります。
また,古典論理の下ではこれらが全て等価になることは帰納法によって容易に確かめられます
%\footnote{本稿全体の議論を踏まえると,古典論理に於いて排中律が成り立つという事実の系として導くことができます。}
。

では,これらの逆はどうでしょうか?
また,Kuratowski有限性と部分有限性との関係も問題たりえます。
実は構成的数学の文脈だとこれらの成立如何を決定することができないことが知られており,
この証明を次の副節以降の目標とします。
構成的数学の文脈でこれを確かめる上では,
必ずしも証明できるとは限らない命題をよく知られた命題と関連付け,
その命題の独立性に帰着させる手法が取られます。
我々の議論では排中律に注目することとし,
証明の中では決定可能等価性や限定全知原理なる概念を必要とします。
そこで次の副節では部分有限集合と決定可能等価性との関連から見ていきます。

\subsection{部分有限集合および決定可能等価性}

本副節では,決定可能等価性を用いて有限性の関係を調べます。
まずは何はともあれ定義をしましょう。

\begin{definition}
    集合\(\Set\)が決定可能等価性(英:decidable equality)を満たすとは,
    集合\(\Set\)の任意の二元\(\Element\),\(\Element<y>\)につき,
    \((\Element=\Element<y>)\OR\NOT(\Element=\Element<y>)\)が成り立つことをいう。
\end{definition}
一般に,\(\Proposition\OR\NOT\Proposition\)を満たす命題\(\Proposition\)を決定可能命題(英:decidable proposition)といいます。
排中律は,任意の命題が決定可能であるということですから,
特に決定可能等価性は,排中律が成り立つ下では常に成立します。
よって古典論理では任意の集合が決定可能等価性を満たすことが分かりますが,
構成的数学に於ける例として,次が挙げられます。

\begin{proposition}\label{prop::all_subfinite_has_decidable_equality}
    部分有限集合は決定可能等価性を満たす。
\end{proposition}

\begin{proof}
    部分有限集合\(\Set\)について,
    \(\Set\)の部分有限性の証拠子\(\FunctionStyle{\Map}{\Set}{\NaturalNumber}\)を取る。
    二元\(\Element\),\(\Element<y>\)を任意にとり,\(\Map(\Element)\),\(\Map(\Element<y>)\)について考えると,
    \(\NaturalNumber\)は決定可能等価性を満たすので,次が成り立つ:
    \begin{center}
        \((\Map(\Element)=\Map(\Element<y>))\OR\NOT(\Map(\Element)=\Map(\Element<y>))\)
    \end{center}
    よって,\(\Map\)の単射性より\((\Element=\Element<y>)\OR\NOT(\Element=\Element<y>)\)が帰結できる。
    これが示したかったことである。
\end{proof}

この系として,Bishop有限集合は決定可能等価性を満たすことが分かります。
その他にも,次のような例があります。{\color{red}[加筆予定]}
また,この証明の議論から部分有限性が部分集合を取る操作について閉じていることが分かります。
Bishop有限集合が決定可能等価性を満たすことを先に帰納法で示せば,
この二つから\cref{prop::all_subfinite_has_decidable_equality}を導くこともできます。
ここで,Kuratowski有限集合が決定可能等価性を満たすか否かを考察すると,
排中律との関係が明らかになります。

\begin{proposition}\label{prop::all_kuratowski_finite_has_not_decidable_equality}
    「任意のKuratowski有限集合が決定可能等価性を満たす」ならば,排中律が成り立つ。
\end{proposition}

\begin{proof}
    命題\(\Proposition\)を任意にとる。
    \(\NaturalNumber<2>=\Set{0,1}\)上の同値関係\(\sim\)を
    \begin{center}
        \(\Element\sim\Element<y> \Longleftrightarrow (\Element=\Element<y>)\OR\Proposition\)
    \end{center}
    とする。
    このとき商集合への標準的射影\(\FunctionStyle{\Map<\pi>}{\NaturalNumber<2>}{\QuotientStyle{\NaturalNumber<2>}{\sim}}\)が構成でき,
    これは全射であるため,\(\QuotientStyle{\NaturalNumber<2>}{\sim}\)はKuratowski有限である。
    よって仮定から決定可能等価性を満たし,
    \begin{center}
        \(([0]\sim[1])\OR\NOT([0]\sim[1])\)
    \end{center}
    が成り立つ。
    いま,上式の部分論理式\([0]\sim[1]\)は定義より\((0=1)\OR\Proposition\),即ち,\(\Proposition\)と同値であるから,
    上式は\(\Proposition\OR\NOT\Proposition\)と書き換えられる。
    これが示したかったことである。
\end{proof}

ここで\cref{prop::all_subfinite_has_decidable_equality}と\cref{prop::all_kuratowski_finite_has_not_decidable_equality}とを比較すると,
二つの有限性の関係に関する主張が二つ得られます。
この主張から,構成的数学ではKuratowski有限集合が部分有限性を満たすことは証明できないことが分かります。

\begin{proposition}\label{prop::kuratowski_is_not_subfinite}
    「任意のKuratowski有限集合が部分有限性を満たす」ならば排中律が成り立つ。
\end{proposition}

\begin{proof}
    仮定と\cref{prop::all_subfinite_has_decidable_equality}とを組み合わせると,
    「任意のKuratowski有限集合が決定可能等価性を満たす」ことが従う。
    更に\cref{prop::all_kuratowski_finite_has_not_decidable_equality}より,
    排中律が成り立つことが分かる。
\end{proof}

\begin{proposition}\label{prop::kuratowski_is_not_bishop}
    「任意のKuratowski有限集合がBishop有限性を満たす」ならば排中律が成り立つ。
\end{proposition}

\begin{proof}
    Bishop有限集合は部分有限であるから,仮定より「任意のKuratowski有限集合が部分有限性を満たす」が成り立つ。
    \cref{prop::kuratowski_is_not_subfinite}より,排中律が成り立つことが分かる。
\end{proof}

\subsection{Kuratowski有限集合と限定全知原理}

本副節では,前副節までで解決できずに残されていた問題「任意のKuratowski有限集合は部分有限か」について考察します。
限定全知原理を用いるため,これの定義から始めます。
なお,命題の間の論理的な関係性は前副節と同様の流れで進むため,
定義の後は淡々と証明を続けることにしましょう。

\begin{definition}
    集合\(\Set\)が限定全知原理(英:limited principle of omniscence)を満たすとは,
    任意の写像\(\FunctionStyle{\Map}{\Set}{\NaturalNumber<2>}\)につき,
    次の二条件の選言が成り立つことをいう:
    \begin{enumerate}
        \item 集合\(\Set\)の要素\(\Element\)が存在し,\(\Map(\Element)=\NaturalNumber<1>\)が成り立つ。
        \item 集合\(\Set\)の任意の要素\(\Element\)につき,\(\Map(\Element)=\NaturalNumber<0>\)が成り立つ。
    \end{enumerate}
    この命題を\(\LPO_{\Set}\)と書く。
    \(\Set\)に関する限定全知原理が成り立つとき,
    集合\(\Set\)は探索可能(英:searchable)であるという。
\end{definition}

\begin{proposition}\label{prop::all_kuratowski_finite_has_searchability}
    任意のKuratowski有限集合は探索可能である。
\end{proposition}

\begin{proof}
    Kuratowski有限集合\(\Set\)について,
    Kuratowski有限性の証拠子\(\FunctionStyle{\Map}{\NaturalNumber}{\Set}\)を取る。
    写像\(\FunctionStyle{\Map<g>}{\Set}{\NaturalNumber<2>}\)を任意にとると,
    合成写像\(\Map<g>\circ\Map\)は\(\NaturalNumber\)から\(\NaturalNumber<2>\)への写像であり,
    Bishop有限集合が探索可能であることから次の二条件の選言が成り立つ:
    \begin{enumerate}
        \item 集合\(\NaturalNumber\)の要素\(\Element\)が存在し,\(\Map<g \circ f>(\Element)=\NaturalNumber<1>\)が成り立つ。
        \item 集合\(\NaturalNumber\)の任意の要素\(\Element\)につき,\(\Map<g \circ f>(\Element)=\NaturalNumber<0>\)が成り立つ。
    \end{enumerate}
    一つ目の条件は「集合\(\NaturalNumber\)の要素\(\Element\)が存在し,\(\Map<g>(\Element)=\NaturalNumber<1>\)が成り立つ」を導き,
    二つ目の条件は\(\Map\)の全射性より「集合\(\NaturalNumber\)の任意の要素\(\Element\)につき,\(\Map<g>(\Element)=\NaturalNumber<0>\)が成り立つ」を導く。
    これは示したかったことである。
\end{proof}

この系として,Bishop有限集合は探索可能であることが分かります。
その他にも,次のような例があります。{\color{red}[加筆予定]}
また,この証明の議論から探索可能性が像を取る操作について閉じていることが分かります。
Bishop有限集合が探索可能であることを先に帰納法で示せば,
この二つから\cref{prop::all_kuratowski_finite_has_searchability}を導くこともできます。

\begin{proposition}\label{prop::all_subfinite_has_not_searchability}
    「任意の部分有限集合が探索可能である」ならば排中律が成り立つ。
\end{proposition}

\begin{proof}
    命題\(\Proposition\)を任意にとる。
    いま,集合\(\Set<P>\)を\(\Set<P>\coloneqq\Set{\Element\in\NaturalNumber<1>|\Proposition}\)と定義すると,
    定義から単射写像\(\Set<P>\to\NaturalNumber<1>\)が存在するので\(\Set<P>\)は部分有限である。
    よって仮定から探索可能であること,即ち,任意の写像\(\FunctionStyle{\Map}{\Set}{\NaturalNumber<2>}\)につき,
    \begin{center}
        \(
            (\exists\Element\in\Set<P>[\Map(\Element)=1])
            \OR (\forall\Element\in\Set<P>[\Map(\Element)=0])
        \)
    \end{center}
    が成り立つことが分かる。
    ここで写像\(\FunctionStyle{\Map}{\Set<P>}{\NaturalNumber<2>}\)を\(\Map(x)=\NaturalNumber<1>\)と定めると,
    \begin{center}
        \(
            \exists\Element\in\Set<P>[\Map(\Element)=1]
            \Longleftrightarrow \exists\Element\in\NaturalNumber<1>[\Proposition]
            \Longleftrightarrow \Proposition
        \)
    \end{center}
    であり,系として\(\forall\Element\in\Set<P>[\Map(\Element)=0] \Longleftrightarrow \NOT\Proposition\)が従う。
    \(\Set<P>\)の探索可能性は命題\(\Proposition\)を用いて\(\Proposition\OR\NOT\Proposition\)と換言され,
    これが示したかったことである。
\end{proof}

\begin{proposition}\label{prop::subfinite_is_not_kuratowski}
    「任意の部分有限集合がKuratowski有限性を満たす」ならば排中律が成り立つ。
\end{proposition}

\begin{proof}
    仮定と\cref{prop::all_kuratowski_finite_has_searchability}とを組み合わせると,
    「任意の部分有限集合が探索可能性を満たす」ことが従う。
    更に\cref{prop::all_subfinite_has_not_searchability}より,
    排中律が成り立つことが分かる。
\end{proof}


\begin{proposition}\label{prop::subfinite_is_not_bishop}
    「任意の部分有限集合がBishop有限性を満たす」ならば排中律が成り立つ。
\end{proposition}

\begin{proof}
    Bishop有限集合はKuratowski有限であるから,仮定より「任意の部分有限集合がKuratowski有限性を満たす」が成り立つ。
    \cref{prop::subfinite_is_not_kuratowski}より,排中律が成り立つことが分かる。
\end{proof}

\subsection{三つの有限性の関係と排中律の等価性}
前副節までで三つの有限性の関係については非自明なものが得られないことが分かりました。
この観察を多少進めると,実は逆が成り立つことも分かります。

\begin{proposition}\label{prop::characterize_lem}
    次の条件は等価である。
    \begin{enumerate}
        \item\label{prop::characterize_lem::item_1} 
            排中律が成り立つ。
        \item\label{prop::characterize_lem::item_2}  
            任意の集合が決定可能性を満たす。
        \item\label{prop::characterize_lem::item_3}  
            任意のKuratowski有限集合が決定可能等価性を満たす。
        \item\label{prop::characterize_lem::item_4}
            任意の集合が探索可能である。
        \item\label{prop::characterize_lem::item_5}  
            任意の部分有限集合が探索可能である。
        \item\label{prop::characterize_lem::item_6}  
            任意のKuratowski有限集合がBishop有限性を満たす。
        \item\label{prop::characterize_lem::item_7}  
            任意のKuratowsiki有限集合が部分有限性を満たす。
        \item\label{prop::characterize_lem::item_8}  
            任意の部分有限集合がBishop有限性を満たす。
        \item\label{prop::characterize_lem::item_9}  
            任意の部分有限集合がKuratowski有限性を満たす。
    \end{enumerate}
\end{proposition}

\begin{proof}
    \cref{prop::characterize_lem::item_1}より\((x=y)\OR\NOT(x=y)\)が示されるので,
    \cref{prop::characterize_lem::item_2}が従う。
    \cref{prop::characterize_lem::item_2}ならば\cref{prop::characterize_lem::item_3}もよく,
    \cref{prop::characterize_lem::item_3}ならば\cref{prop::characterize_lem::item_1}は\cref{prop::all_kuratowski_finite_has_not_decidable_equality}よりよい。

    \cref{prop::characterize_lem::item_1}より\cref{prop::characterize_lem::item_4}が従うことも同様によく,
    \cref{prop::characterize_lem::item_4}ならば\cref{prop::characterize_lem::item_5}は明白,
    \cref{prop::characterize_lem::item_5}ならば\cref{prop::characterize_lem::item_1}は\cref{prop::all_subfinite_has_not_searchability}よりよい。

    \cref{prop::characterize_lem::item_6}ならば\cref{prop::characterize_lem::item_7}は明白であり,
    \cref{prop::characterize_lem::item_7}ならば\cref{prop::characterize_lem::item_1}は\cref{prop::kuratowski_is_not_subfinite}よりよく,
    \cref{prop::characterize_lem::item_8}ならば\cref{prop::characterize_lem::item_9}は明白であり,
    \cref{prop::characterize_lem::item_9}ならば\cref{prop::characterize_lem::item_1}は\cref{prop::subfinite_is_not_kuratowski}よりよい。

    よって,
    \cref{prop::characterize_lem::item_1}ならば\cref{prop::characterize_lem::item_6}および
    \cref{prop::characterize_lem::item_1}ならば\cref{prop::characterize_lem::item_8}を示せば充分であるが,
    これは既に定義の直後に述べている{\color{red}[加筆予定: 帰納法が回るとしか書いていないので]}。
\end{proof}
