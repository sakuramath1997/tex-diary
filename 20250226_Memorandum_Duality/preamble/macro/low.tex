\newcommand{\DefineWord}[2]{\expandafter\newcommand\csname Word#1\endcsname{#2}}

\usepackage{expl3}
% meta-macros
\NewDocumentCommand{\Placeholder}{s}{ \IfBooleanTF{#1}{ ? }{ \text{\textendash} } }
\newcommand{ \SetMiddle }{ \mathrel{}\middle|\mathrel{} }
\NewDocumentCommand{\ParenLR}{ d() }{ \left( #1 \right) }
\ExplSyntaxOn
\NewDocumentCommand{\CenterMid}{m}
{
    \seq_set_split:Nnn \l_tmpa_seq {|}{#1}
    \seq_if_empty:NTF \l_tmpa_seq
    {
        #1
    }
    {
        \seq_pop_left:NN \l_tmpa_seq \l_tmpa_tl
        \l_tmpa_tl
        \seq_if_empty:NF \l_tmpa_seq
        {
            \seq_pop_left:NN \l_tmpa_seq \l_tmpa_tl
            \expandafter\SetMiddle\l_tmpa_tl
        }
        \seq_map_inline:Nn \l_tmpa_seq
        {
            \expandafter\SetMiddle##1\\
        }
    }
}

% HelperFunction: Apply _{index} to each comma-separated element
\NewDocumentCommand{\HelperFunction}{m m}
{
     \clist_map_variable:NNn #1 \l_tmpa_clist
    {
    \csname MyFunc\endcsname{##1}{#2} % 変換処理
    \clist_if_empty:NF \l_tmpa_clist { , } % 次の要素が存在すればカンマを追加
    %\tl_if_empty:NF \l_tmpa_tl { , } % 次の要素が空でなければカンマを追加
    }
}
%{
% \clist_map_inline:nn {#1}
%  {
%   \csname MyFunc\endcsname{##1}{#2} % 変換処理
%   %\clist_if_in:NnTF {#1} {##1} {, } {} % 最後の要素でなければカンマを追加
%   \tl_if_empty:NF \l_tmpa_tl {, } 
%  }
%  %\unskip
%}
%% {
%%  % 引数リストを分割して処理
%%  \clist_map_inline:nn {#1} { \csname MyFunc\endcsname{##1}{#2}, }
%%  % 最後の余分なカンマを削除するために余分なスペースを削る
%%  \unskip
%% }

\ExplSyntaxOff
\NewDocumentCommand{ \ConceptDefinitor }{ m d<> t{'} e{_} d() e{_} }{ {\IfValueTF{#2}{#2}{#1}}\IfValueT{#4}{_{#4}}\IfBooleanT{#3}{'}\IfValueT{#5}{\ParenLR(#5)}\IfValueT{#6}{_{#6}} }
%\NewDocumentCommand{\HOGE}{d<>}{\ConceptDefinitor{X}<#1>}
\NewDocumentCommand{ \MapDefinitor }{ m d<> t{'} e{_} d() e{_} }{ {\IfValueTF{#2}{#2}{#1}}\IfValueT{#4}{_{#4}}\IfBooleanT{#3}{'}\ParenLR(\IfValueTF{#5}{#5}{\Placeholder})\IfValueT{#6}{_{#6}} }




% General Usage
\NewDocumentCommand{\Object}{d<>}{\ConceptDefinitor{x}<#1>}

% Set Theory
\NewDocumentCommand{\Set}{ g d<> }{
	\IfValueTF{#1}{
		\left\{\CenterMid{#1}\right\}
	}{
		\IfValueTF{#2}{
			#2
		}{
			X
		}
	}
}
% Category Theory
% Higher Category Theory
\NewDocumentCommand{\InfinityCategory}{d<>}{\ConceptDefinitor{\mathcal{C}}<#1>}
\NewDocumentCommand{\IndexedInfinityCategory}{d<>}{\InfinityCategory<\IfValueTF{#1}{#1}{I}>}
\NewDocumentCommand{\OnePointInfinityCategory}{}{\InfinityCategory<\ast>}

\NewDocumentCommand{\InfinityFunctor}{d<>}{\ConceptDefinitor{F}<#1>}